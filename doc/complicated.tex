\begin{comment}
/*
\end{comment}
\section{A considerably more complicated 2D example.}
\begin{comment}
*/
\end{comment}

This example demonstrates a lot more of what you can do using dactyl.  The
system is the same as in the previous example, but this time we will
calculate the quality factor of the cavity.  Again, the entire control file
will be included here, but I'll skip over sections that have already been
explained.

\begin{verbatim}
#include "dactyl.h"

const double half_cavity_width = 0.5*0.68, air_slit_width = 0.38,
             grating_periodicity = 0.48,
             half_waveguide_width = 1.0,
             num_air_slits = 15.0,
             high_dielectric = 12.0, low_dielectric = 11.5;
const double pml_thickness = 1.0;
const double x_center = 7.7 + pml_thickness;
const double y_center = 10.5 + pml_thickness;
double eps(const vec &rr) {
  // Displacement from center of cavity is r:
  const vec r = rr - vec2d(x_center, y_center);
  // First the air slits:
  double dx = fabs(r.x()) - half_cavity_width;
  if (dx < num_air_slits*grating_periodicity && dx > 0.0) {
    while (dx > grating_periodicity) dx -= grating_periodicity;
    if (dx < air_slit_width) return 1.0;
  }
  // Now check if the y value is within the waveguide:
  if (fabs(r.y()) < half_waveguide_width) return high_dielectric;
  // Otherwise we must be in the surrounding low dielectric:
  return low_dielectric;
}
\end{verbatim}
This time we use the \verb!deal_with_ctrl_c();! function.  This is a handy
utility function that is useful when running your dactyl code
interactively.  It traps the SIGINT signal, so when you hit cntl-C, rather
than simply exiting, the global variable \verb!interrupt! is incremented.
If you really want to exit, just hit cntl-C again, and when
\verb!interrupt! reaches 2, the program will exit.
\begin{verbatim}
int main(int argc, char *argv[]) {
  deal_with_ctrl_c();
  initialize(argc, argv);
  const double amicron = 10; // a micron is this many grid points.
  const symmetry S = mirror(Y, voltwo(2*x_center, 2*y_center, amicron));
  mat ma(voltwo(2*x_center, 2*y_center, amicron), eps, 0, S);
  ma.use_pml_everywhere(pml_thickness);
  const char *dirname = make_output_directory(argv[0]);
  ma.set_output_directory(dirname);
  fields f(&ma);
  const double wavelength = 1.72;
  const double freq = 1.0/wavelength;
  f.add_point_source(Hy, freq, 5.0, 0.0, 5.0, vec2d(x_center,y_center));
  f.use_metal_everywhere();
\end{verbatim}
We add an additional check below ``\verb!&& !interrupt!'' so that when the
user hits cntl-C we exit the loop.
\begin{verbatim}
  while (f.time() < f.find_last_source() && !interrupt) f.step();
  f.eps_slices();
  while (f.time() < 400.0 && !interrupt) f.step();
\end{verbatim}
This time we're going to run the simulation longer, so we would like to get
occasional informative messages.  To do this we define a variable to hold
the next time we want to print a message.
\begin{verbatim}
  double next_print_time = 500.0;
\end{verbatim}
To calculate the $Q$ of our cavity, we use a monitor point \verb!p!.  We
also store the value of $H_y$ at our monitor point in a file named ``hy''
periodically.  We create this file using \verb!create_output_file!, which
creates and opens a file in the given output directory.  This utility
function really isn't what we want here, but I haven't yet written one
which is analagous to \verb!master_printf! in that you will only get one
copy of each thing you print to the file.
\begin{verbatim}
  monitor_point *p = NULL;
  FILE *myout = create_output_file(dirname, "hy");
  while (f.time() <= 2000.0 && !interrupt) {
    // Now we'll start taking data!
    f.step();
\end{verbatim}
To get the monitor point data we use the \verb!get_new_point! method of
\verb!fields!.  This ends up creating a linked list containing the values
of the field at the monitor point as a function of time, which we will
later use to run harminv and get the $Q$.
\begin{verbatim}
    p = f.get_new_point(vec2d(x_center,y_center), p);
\end{verbatim}
We use the \verb!get_component! method to extract the (complex) fields from
the monitor point so we can print them.
\begin{verbatim}
    fprintf(myout, "%lg\t%lg\t%lg\n", f.time(),
            real(p->get_component(Hy)),
            imag(p->get_component(Hy)));
\end{verbatim}
Every time we reach the \verb!next_print_time! we print out a copy of the
slices, along with a little message indicating the time and the total
energy (which should be decaying at this point).  The function
\verb!master_printf! is a utility function that works basically like
\verb!printf!, except that when running in parallel only one of the
processors (the ``master'') does the printing.  You should use this
function rather than something like
``\verb!if (my_rank()==0) printf(...)!'', since the latter can cause
problems if the arguments to printf require synchronization between the
processes.
\begin{verbatim}
    if (f.time() == next_print_time) {
      f.eps_slices();
      master_printf("Energy is %lg at time %lg\n",
                    f.total_energy(), f.time());
      next_print_time += 500.0;
    }
  }
  fclose(myout);
\end{verbatim}
Having collected all the monitor point data, we now want to run
harminv\footnote{If you don't know what harminv is, I'm not going to
explain it here, so you may as well ask me in person (or even better, ask
Steven...} on it to find the $Q$ of our resonant cavity.  The
\verb!harminv! method gives us the complex amplitudes, the frequencies and
the decay rates.  The decay rate is given in the same units as the
frequency, so you could choose to view it as the imaginary part of the
frequency if you like.

This harminv step is the real reason for using the ctrl-C trick, since if
while running this example we get impatient and decide we have enough data
we can just hit ctrl-C and get the results using what data we have.  This
means you can just set the code to run for an excessively long time without
risking losing everything if you lose patience.
\begin{figure}
\label{complicated_figure}
\caption{Contents of ``freqs'' file}
\input{complicated-out/freqs}
\end{figure}

In case you're wondering about the ``\verb!\begin{verbatim}!'', it's there
so I can easily include the output in this manual (see
Figure~\ref{complicated_figure}).
\begin{verbatim}
  complex<double> *amp;
  double *freqs, *decays;
  int num;
  
  FILE *myfreqs = create_output_file(dirname, "freqs");
  fprintf(myfreqs, "\\begin{verbatim}\n");
  master_printf("Harminving Hy...\n");
  interrupt = 0; // Harminv even if we were interrupted.
  p->harminv(Hy, &amp, &freqs, &decays, &num, 0.8*freq, 1.2*freq, 5);
  for (int i=0;i<num;i++) {
    fprintf(myfreqs, "%lg\t%lg\t%lg\t%lg\t%lg\n",
            freqs[i], decays[i], -freqs[i]/decays[i],
            real(amp[i]), imag(amp[i]));
  }
  fprintf(myfreqs, "%cend{verbatim}\n", '\\');
  fclose(myfreqs);
  delete[] dirname;
  finished();
}
\end{verbatim}
